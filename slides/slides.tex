% vim: set et ts=4 sw=4:
\documentclass[a4paper,12pt,presentation]{beamer}

\mode<presentation>{
    \usetheme{llbit}
    \setbeamercovered{transparent}
}

\usepackage{fancyvrb}

% input
\usepackage[utf8]{inputenc}
\usepackage[english]{babel}

% typefaces
\usepackage[T1]{fontenc}
\usepackage[scaled]{helvet}
\usepackage[scaled]{beramono}

\title[tinytemplate]{A tiny template engine}
\author{Jesper Öqvist}
\institute{Department of Computer Science\\
Lund University}

\begin{document}

\begin{frame}
    \titlepage
\end{frame}

%\begin{frame}{Outline}
%    \tableofcontents
%\end{frame}

\begin{frame}
    \frametitle{Why Tiny Templates?}
    \begin{itemize}

        \item A template system should be as simple as possible

        \item Control statements make templates harder to read, edit, and test

        \item If the template syntax is kept simple, the template
            system will be easy to use

    \end{itemize}
\end{frame}

\begin{frame}
    \frametitle{What we Need}
    \begin{itemize}

        \item Simple templates

            \pause

            \begin{itemize}
                \item Only allow variable (and attribute) expansion in templates
            \end{itemize}

        \pause

        \item Intelligent indentation

            \pause
            \begin{itemize}
                \item Customizable output indentation
                \item Variable expansions are indented per line
            \end{itemize}

        \pause

        \item Multiple templates per template file (convenient)

    \end{itemize}
\end{frame}

\begin{frame}[fragile]
    \frametitle{Variable Expansion}

    \begin{itemize}
        \item Variable references have two forms:
            \begin{itemize}
                \item \verb'$varname'
                \item \verb'$(varname)'
            \end{itemize}
        \item Variable names can include periods, but remember to use
            parenthesis: \verb'$(var.name)'
        \item Parenthesis are also needed when the reference
            is in the middle of a word: \verb'$(var)_name'
    \end{itemize}
    
\end{frame}

\begin{frame}[fragile]
    \frametitle{Context}

    \begin{itemize}
        \item All template expansions happen with a given \emph{context}
        \item The context is an ordinary Java \verb'Object'
        \item Use \verb'pushContext' and \verb'popContext' in the template
            expanding Java code
    \end{itemize}
    
\end{frame}

\begin{frame}[fragile]
    \frametitle{Attribute Expansion}

    \begin{itemize}
        \item A hash sign is used to reference an attribute:
            \begin{itemize}
                \item \verb'#name'
                \item \verb'#(name)'
            \end{itemize}
        \item Attributes are Java methods invoked on the context object
        \item The attribute method must take no arguments and have non-void
            return type
        \item The expansion of an attribute is equal to the return value
            of the attribute method converted to a string
    \end{itemize}
    
\end{frame}

\begin{frame}[fragile]
    \frametitle{Template Declarations}

    \begin{itemize}
        \item A template file includes a number of template declarations
        \item Each template is declared with one or more names
        \item Template bodies are enclosed in \verb'[[' and \verb']]'
        \item The \verb'=' is optional
    \end{itemize}

    Examples:
    \begin{verbatim}
        template1 = [[ ]]

        template2 = [[xy]]

        X = Y = [[$x]]
    \end{verbatim}
\end{frame}

\begin{frame}[fragile]
    \frametitle{JastAdd Integration}

    \begin{verbatim}
        // Test.jrag
        syn String A.name() = "myName";

        // Test.java
        tt.pushContext(new A());
        tt.expand("my.template", out);
        tt.popContext();

        // Test.tt
        my.template = [[public String #name;]]

        // out
        public String myName;
    \end{verbatim}
\end{frame}

\end{document}

