% vim: set et ts=4 sw=4:
\documentclass[a4paper,12pt,presentation]{beamer}

\mode<presentation>{
    \usetheme{llbit}
    \setbeamercovered{transparent}
}

\usepackage{fancyvrb}

% input
\usepackage[utf8]{inputenc}
\usepackage[english]{babel}

% typefaces
\usepackage[T1]{fontenc}
\usepackage[scaled]{helvet}
\usepackage[scaled]{beramono}

\title[tinytemplate]{tinytemplate}
\author{Jesper Öqvist}
\institute{Department of Computer Science\\
Lund University}

\begin{document}

\begin{frame}
    \titlepage
\end{frame}

%\begin{frame}{Outline}
%    \tableofcontents
%\end{frame}

\begin{frame}
    \frametitle{Goals}
    \begin{itemize}

        \item Simple syntax

        \item A minimal feature set

        \item Thorough test suite

        \item Intelligent indentation

            \pause
            \begin{itemize}

                \item Customizable output indentation

                \item Each line of an expanded string should be properly
                    indented

            \end{itemize}

    \end{itemize}
\end{frame}

\begin{frame}
    \frametitle{Features}
    \begin{itemize}

        \item Multiple templates per template file

        \item Variable expansion

        \item Attribute expansion

        \item List concatenation

        \item Conditional expansion

        \item Subtemplate expansion

    \end{itemize}
\end{frame}

\begin{frame}[fragile]
    \frametitle{Template Declarations}

    \begin{itemize}
        \item A template file includes a number of template declarations
        \item Each template is declared with one or more names
        \item Template bodies are enclosed in \verb'[[' and \verb']]'
        \item The \verb'=' is optional
    \end{itemize}

    Examples:
    \begin{verbatim}
        template1 = [[ ]]

        template2 = [[xy]]

        X = Y = [[$x]]
    \end{verbatim}
\end{frame}

\begin{frame}[fragile]
    \frametitle{Context}

    \begin{itemize}
        \item All template expansions happen with a \emph{context} object
        \item The context is used to lookup variables and attributes
        \item \verb'SimpleContext' is the standard context - it stores variables
            in a map and evaluates attributes by invoking methods on an object
    \end{itemize}
    
\end{frame}

\begin{frame}[fragile]
    \frametitle{Variable Expansion}

    \begin{itemize}
        \item Variable references have two forms:
            \begin{itemize}
                \item \verb'$varname'
                \item \verb'$(varname)'
            \end{itemize}
        \item Variable names can include periods, but remember to use
            parenthesis: \verb'$(var.name)'
        \item Parenthesis are also needed when the reference
            is in the middle of a word: \verb'$(var)_name'
        \item {\bf A Java identifier is parsed if no parenthesis are present.}
    \end{itemize}
    
\end{frame}

\begin{frame}[fragile]
    \frametitle{Attribute Expansion}

    \begin{itemize}
        \item A hash sign is used to reference an attribute:
            \begin{itemize}
                \item \verb'#name'
                \item \verb'#(name)'
            \end{itemize}
        \item Attributes are Java methods invoked on the context object
            (SimpleContext)
        \item The attribute method must take no arguments and have non-void
            return type
        \item The expansion of an attribute is the return value of the
            attribute method converted to a string
    \end{itemize}
    
\end{frame}

\begin{frame}[fragile]
    \frametitle{List Concatenation}
    \begin{itemize}
        \item Concatenates a \verb'Iterable' list of objects using an optional
            separator string
        \item Examples (without \& with separator):
            \begin{itemize}
                \item \verb'$cat(#list)' \\
                    Expands to ``123''
                \item \verb'$cat(#list, ", ")' \\
                    Expands to ``1, 2, 3''
            \end{itemize}
    \end{itemize}
\end{frame}

\begin{frame}[fragile]
    \frametitle{Conditional Expansion}
    \begin{itemize}
        \item Examples (without \& with separator):
            \begin{itemize}
                \item \verb'$if(a) hello $endif' \\
                    Expands to ``hello'' if
                    the variable a expands to ``true''
                \item \verb'$if(a) hello $else goodbye $endif' \\
                    Expands to
                    ``hello if a is ''true``, otherwise ''goodbye``
                \item \verb'$if(#a) hello $endif' \\
                    If the {\bf attribute} a
                    evaluates to ''true`` the statement expands to ''hello``
            \end{itemize}
    \end{itemize}
\end{frame}

\begin{frame}[fragile]
    \frametitle{Subtemplate Expansion}
    \begin{itemize}
        \item Templates can be expanded inside each other (recursion will
            result in a stack overflow!):
            \begin{itemize}
                \item \verb'$include(a)' \\
                    Expands the template named ``a'' in the current template.
                    Indentation works as expected -- if the statement is
                    indented the expansion will be indented the same additional
                    amount.
            \end{itemize}
    \end{itemize}
\end{frame}

\begin{frame}[fragile]
    \frametitle{JastAdd Integration}

    \begin{verbatim}
        // Test.jrag:
        syn String A.name() = "myName";

        // Test.java:
        tt = templateFromFile("Test.tt");
        context = new SimpleContext(tt, new A());
        context.expand("my.template", out);

        // Test.tt:
        my.template = [[public String #name;]]

        // output:
        public String myName;
    \end{verbatim}
\end{frame}

\end{document}

